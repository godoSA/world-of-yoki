\documentclass[10pt,a4paper]{article}
\usepackage[utf8]{inputenc}
\usepackage{amsmath}
\usepackage{amsfonts}
\usepackage{amssymb}
\usepackage{graphicx}
\usepackage{hyperref}
\usepackage{textcomp}
\usepackage[final]{pdfpages}
\usepackage{fancyhdr}
\usepackage{makeidx}
\usepackage{multicol}
\usepackage{tabto}
\fontencoding{T1}
\pagestyle{fancyplain}
\author{}
\title{YOKI - fucking mad}
\date{}
\makeindex
\hypersetup{
	colorlinks,
	citecolor=black,
	filecolor=black,
	linkcolor=black,
	urlcolor=black
}
\setlength{\columnsep}{1cm}
\setlength{\columnseprule}{1pt}
\begin{document}
	\maketitle
	\pagebreak
	\setcounter{tocdepth}{3}
	\tableofcontents
	\pagebreak
The Cursed
The blood of the cursed take three main caste: The Turned, The Blessed and
The Eerie. They see each other as kin, though there are some noticeable
differences.
When embracing a new childer, he does not necessarily turn out the same
clan; while the caste is the same.
This does not account for the turned, as their clans are more of a way, or
tradition of teachings, than actual difference of the curse. Which in turn
means that a childer of a Turned could be stolen from him, and tought by
the other clan and make little difference, clan wise.
A childer of any other caste is first raised by his sire, and meanwhile
counted as a member of the clan of his sire. Upon becoming a neonate, the
vampire also assumes membership of his true clan.
All Kindred has in common the ability to hastily heal for a period of time
after feeding on living human blood.The Turned
Sometime, in ages long forgotten; one of the damned caste forced his power
into the body of the unborn, and the childe born to become a monster,
powerful in using the life force of their follow human. Not many years after
being born, he began wrecking havoc and destruction over all he could see.
Though one day he felt lonely, and forced his power unto a living girl; the
human became as him, save some of the power. They campaigned the world
together, but eventually they split paths. He saw no reason to destroy her,
since he was not yet bored this day, and thus let her live. She roamed the
world, repenting over her previous actions and found her only way to
redemption to destroy her master. As she knew she was no match for him,
she went to a king, who's land was terrorized by her master, and asked for
his assistance in building an army; after showing her power the king agreed.
Together they brought together the bodies of the many dead throughout the
kingdom. Using the power of the dead blood, she started turning scholars
and soldiers into a versatile force. United they attacked her master, and after
an exhausting battle her master had been slain. She took the few of the
survivors of the battle and started the house Tremere.
What she did not know was that during the battle she had been pregnant;
and her child had been imbued by her masters powers. The childe was borne
and they named him Ravnos, and not long after they all noticed the power
of the child, though they decided to not kill it; instead they decided to make
an attempt to raise it, and use the child's power. When the child neared 14,
none of the house Tremere could match his power, and he used it to free the
slaves of the house, for they had succeeded well in raising him with good
morals and he saw the evil of slavery and the cruelty displayed by his peers.
After having struck down all who opposed the freeing of the slaves; he
ventured away with the slaves to build his own tribe.
The turned continue to grow old when not under the influence of another
humans blood; the ageing is paused again as soon as the blood touch their
lips, if the blood is drawn directly from the vein.
The act of creating a member of this path can take two forms, either the
traditional way, or the ritual way.
The traditional way, performed as the legends tell; an unborn child is
infused with the power of the elders and born a monster, with power far
surpassing those created using the ritual and with a normal lifespan of over
300 years.
The ritual way, performed using the life force of four newborn, not over
three months old, or the combined blood of fifty fully grown men; the
subject of the ritual is imbued with the power of the turned, rendering him
able of powerful magic and longevity.
When one of the Turned are killed, the time catch up to them, quickly
ageing the body and reducing it to dust if old enough.
When staked, they take damage as a normal human would from a stake
through the heart. They are not bothered by sunlight. They can not create
ghouls with the power of their blood, though their blood is stronglyaddictive but not giving a rush effect.
Creation Bonus:
Manipulation +1; Stamina +2; Resolve +1; Intelligence +1
Traditional:
Manipulation +2; Intelligence +2; Wits +2; Strength +1; Stamina +2;
Dexterity +2; Presence +2; Resolve +2; Composure +2
Mild Traditional:
Manipulate +1; Intelligence +1; Wits +1; Resolve +1; Dexterity +1
Ritual:
<None but the default>Tremere
Ritualists and blood mages, wielder of power.
The way of turning most preferred by the Tremere is by the ritual, as many
of the elders of the clan were created using the traditional way; and fear that
if the turning would use that method, they would soon be outnumbered by
younger, powerful Tremere. That result is that most Tremere only know of
the ritual way of turning, and never question it as the only one.
The ruling of house Tremere is strictly Hierarchical, with an elder in charge
of each chantry; a breach of hierarchy is most often frowned upon, even
when necessary or called for.
Disciplines: Auspex; Thaumaturgy; Dominate or Majesty
Views:
• The Turned
The only true Kindred, one day we will show our superiority. But
first the Ravnos need taking care of.
◦ Ravnos
Always scurrying about with their filthy caravans pretending to
be noble. One day we shall have our revenge on them, but now
is not the time.
• The Blessed
Monsters in Disguise, their true face comes out during dinner.
◦ Daeva
They prance around looking like peacocks, seemingly not having
a care in the world. But that's just a trick to lure me in, and soon
they'd have me doing their laundry. And they'd probably have me
beg for it too!
◦ Malkavian
Never know what they might be thinking of, hopefully their
madness wont effect my rituals. Hopefully they won't see me in
their visions. Probably...
◦ Tzimisce
They keep to theirs, leaving me to mine. That's good, more time
for me to prepare for exterminate them like the smut they are.
◦ Ventrue
The only hinder of us ruling all Kindred, though I must admit;
they are superb rulers. But not the rightful rulers.
• The Eerie
Seems human enough, though their true self comes out after a few
days without feeding. The whole bunch of them creep me out.
◦ Gangrel
Bush-mongrels, but they have never bothered me in the city.
They know to keep to themselves. Which is fine for now.
◦ Nosferatu
I wouldn't trust them with taking care of my trash unlessproperly leashed, unless they need something from me.
◦ Lasombra
Probably nothing of value. Though their shadow magic could be
useful, but I don't seem to get the hang of it, I probably have the
ritual all wrong.
• Humans
Nothing more than cattle and storage-boxes of my power.Ravnos
Tricksters, deceivers and thieves, protectors of their tribe.
The Ravnos always performs the turning according to tradition, by choosing
a female of the tribe they bestow upon her the great honour of giving birth
to the next protector. The child is born, and raised under heavy surveillance
of the old protector. So as to not let their monstrous nature get the dominant
hold of the mind. Although they use the traditional way of creation, they
have practised a mild form of it, resulting in less powerful, but still
powerful, childer and most importantly: less murderously insane. That being
said, in some rare cases the mild form is not used and the resulting offspring
suffers the full blow of the traditional birth.
Most Ravnos begins as protectors of their tribe, though when a new
protector is raised, the old is free to stray wherever he might be willing.
This leave some to venture on long travels, some become hermits and yet
some take a go at socialising with other Kindred; though most often they do
this for the amusement value of irritating them.
Disciplines: Auspex or Obfuscate; Celerity: Chimerstry
Views:
• The Turned
It's an honour to be one of the turned, and we are the true Kindred.
◦ Tremere
If only they could forget that old scar, all would be well. Perhaps
I'll make them.
• The Blessed
The whole bunch of them are crazy, but at least they know they're
crazy, and crazy is far better than creepy; and some of them are of
value to us.
◦ Daeva
The Daeva is much like a pretty painting; a blank canvas under
all the paint.
◦ Malkavian
They know something, and they're willing to tell anyone about
it .The only problem is that most won't listen or can't understand
what they are being told.
◦ Tzimisce
Formidable craftsmen; though I would rather not see them at
work.
◦ Ventrue
Lords and leaders, though not my lords and leaders. They will
never order my tribe around. Let them try.
• The Eerie
True to their nature, and they seem more human than the Blessed.
Still, the bunch of them creeps me out, and they seem testy.
◦ GangrelFellow travellers of the roads, many have good travel stories to
tell. Though most Gangrel are often unnecessary violent.
◦ Nosferatu
I've had my run-ins with the Nosferatu, and I've bought bits and
pieces from them, but I seem to have lost my soul somewhere.
◦ Lasombra
They drink their wine. Make them spill and you'd better jump at
shadows.
• Humans
I don't really know that many humans, except for my tribe. But I
guess they're simple enough.The Blessed
Once, in the time when Karanduniash was, one of the damned caste strayed
upon the corpse of a fallen king; and he whispered to the lost soul, asking
him if he wished to walk the world still; and so he did. Alas the damned
helped the king to do this, but even as the spirit had entered the body again
it was still lost. And the King spread the taint upon the world and said he
was a saint, and the taint was not a curse, for it was a blessing.
The longer they go without the feeling of fresh blood, the more human their
behaviour. But when they feel the potency of blood the monster inside them
break loose and take the body in possession, and relish in the mayhem and
bloodlust. The aftershock of the beast can sometimes be felt weeks after the
last supper, making them more physically powerful and less human.
The act of creating a blessed is a rather short ritual: they summon the soul of
a deceased by dripping some of their own blood, and ask if the soul if it
want to come back and if it says yes, so it does. If it says no, the ritual has
failed and can not be attempted again. The creation is not without penalty
though; the spirit suffers such trauma when being forced back that part of
their self is twisted and bent.
Those embraced begin to turn shortly after their creation, straightening
twisted limbs; closing gaping wounds; forming an appearance towards the
now beasts ideals. The body regains awareness within 20 minutes,
movement in 30 and can act as normal within 2 hours.
When one of the Blessed are killed their body dries out and within seconds
turns to a dried out and very brittle husk.
The Blessed are the only caste who experience a pause effect from a
wooden stake through the heart. They are strongly affected by sunlight and
will have trouble moving areas lit by the sun, although shadows offer some
protection. They are perfectly able of creating ghouls, their blood giving an
effect similar to especially addictive cocaine, they are also able to give their
ghouls a great feeling of discomfort on will, if the ghoul still has the
vampires blood in his veins.
Creation Bonus:
Manipulation +1; Presence + 1; Strength +1; Stamina +2Daeva
The high society of the kindred world, snobby and arrogant.
Disciplines: Auspex or Celerity; Majesty; Vigor
Creation Bonus:
Presence +2; Manipulation +1; Dexterity +2
Views:
• The Turned
They seem to believe they are more worth than others; which is
entirely wrong; we are.
◦ Tremere
Practical to have around when you need something done, even if
they do plan on overrunning our society.
◦ Ravnos
Lumps of bad smell. Keep them of the rug, and they'll be on
their way soon enough.
• The Blessed
Sometimes, the blessing do seem like a curse; most of the time I
don't care.
◦ Malkavian
They have the stomach to claim we are all crazed, which must be
wrong. I'm adorable!
◦ Tzimisce
When I first heard they were of the blessed caste I was shocked,
I had always thought they were grotesque beings of the Eerie
caste. Just look at them!
◦ Ventrue
They order people around, and they seem classy enough, and I
am in the need of a new puppet.
• The Eerie
They are filthy, but sometimes it can be useful to have them dance to
my flute.
◦ Gangrel
Pesky animals; once, one of them had the nerve of talking to me
directly!
◦ Nosferatu
I've had my go at haggling with them, I had to sell my Mansion,
and it still seemed like a good deal! Though the knife in my back
was not part of the deal.
◦ Lasombra
Lasombra I can like, they know how to handle themselves in
public places, in differ to most people these days. And they are
marvellous at interior design.
• HumansSimple puppets to use at my whim in whichever way I see fit. But
they do entertain me.Malkavian
Madmens and lunatics; hedge-wizards and charlatans; and, feared by all, the
prophets.
Disciplines: Auspex; Dementation or Dominate; Obfuscate
Sorcery: Theban
Creation Bonus:
Resolve +2; Manipulation +2; Wits +1
Views:
• The Turned
They are all silly gooses.
◦ Tremere
THEY WILL FAIL!
◦ Ravnos
They are nice, they listen to me!
• The Blessed
We are all the blessed, we are all the cursed. We are the same, but
never the same.
◦ Daeva
Oooh, shiny, I would like to have him as part of my collection.
Though their acts as human is all off, according to my studies.
◦ Tzimisce
Could you smooth out my skin, I just had a bath and now I'm all
wrinkly.
◦ Ventrue
No, no, no, noooo... Yes sir; I agree... Damn, he did it again!
• The Eerie
The voices are scary, these are simple eerie.
◦ Gangrel
Who's a good dog, you want me to scratch behind the ear? Yes
you do, yes you do.
◦ Nosferatu
My magic beats your magic.
◦ Lasombra
Please, not there, nooo... phew, just a dream... What's that
shadow doing over there?
• Humans
Even their most mad individual isn't nearly mad enough to see the
truth, but they are worthy of study, as I don't quite remember how it
was to be a human.Tzimisce
Sculptors and artists, with a fascination of flesh, blood and bone.
Disciplines: Animalism or Majesty; Auspex; Vicissitude
Creation Bonus:
Presence +2; Strength +3
Views:
• The Turned
Barely more than humans. Though should not be underestimated..
Although, they do make it difficult.
◦ Tremere
Yes, keep away from me. It will never succeed anyhow.
◦ Ravnos
And people call us monsters.
• The Blessed
We are truly the Blessed kin.
◦ Daeva
They seem to like our art; now, how could I use that against
them?
◦ Malkavian
They make claims we are insane, which might be true to some
extent. However, they are not only insane, they are also
worthless.
◦ Ventrue
Oh, yes, play your little game chess, but I'll be no piece of yours.
• The Eerie
They give me inspiration: inspiration to make more lethal weapons
to kill them with.
◦ Gangrel
Always waiting in the bushes, biding their time for their unjust
revenge. It is us who are the just. Thieving bastards!
◦ Nosferatu
I never could get the hang of the Nosferatu, are they mercenaries
for money or secrets? Anyhow, better not get involved with them
if you want to keep your secrets secret.
◦ Lasombra
I don't fancy meeting one of them in a dark alley, that's why I
crafted a magical torch to put out their shadows.
• Human
Their bodies are easily bent into masterpieces, and their minds are
even easier to bend.Ventrue
The suits and bosses; what they command, you follow.
Disciplines: Dominate or Majesty; Resilience; Vigor
Creation Bonus:
Presence +2; Manipulation +2; Stamina +1
Views:
• The Turned
Powerful sorcery of levels I could never dream of achieving, but all
can be bought.
◦ Tremere
They recognize us as supreme rulers, as they should. Now, we
only need to take care of their attitude.
◦ Ravnos
As for now, they are too much bother. But eventually, you and
your tribe will fall under my reign.
• The Blessed
Some are more blessed then are others. Me in specific.
◦ Daeva
They are under the belief that they can use me, which is fine, as
long as they do what I command.
◦ Malkavian
I'm certain that you would not be fooled again, do you agree?
◦ Tzimisce
When I was young, I was quite the master of board-game chess;
now I have become a master of live-game chess. And I always
need useful pieces.
• The Eerie
Most of them make good soldiers or strategists, and are perfectly
able of taking care of my problems.
◦ Gangrel
He he he, yes.. We will see.
◦ Nosferatu
Always game for a good deal, good for me. I don't know where I
would be without their wish for gold.
◦ Lasombra
They are watchful of the shadows, and for that I respect them,
but they only fear our eyes and ears. They forget about our
hands.
• Humans
As long as they understand to follow my command and jump on my
plate, I have no problem with them.The Eerie
Once, when a hero was walking home from his war, one of the damned
caste captured him; and in many years to follow his body and person was
cruelty tortured. But one day, he was able to slay the damned being.
Though, when walking home, all on which his shadow fell avoided his gaze
and hastily hurried on. After a while he finally came home, though when he
saw his wife, she had taken another man; in a rage, the hero brought
vengeance upon the village that had betrayed him. He no longer was hero;
he was a fallen, a damned and an eerie presence was his constant
companion. The final piece of his vengeance took the form of passing the
damnation to his wife's new man.
When blood has not been drunk, the power of the beast step forward,
making its presence known; resulting in, after a while, massive power in
mostly anything they decide to turn their gaze on. All the while their power
grow, the beast show more and more, making them less of a treat socializing
with. After a sip of blood, they yet again turn as docile as a kitten,
depending on the kitten.
When the Eerie embrace, they empty the body of the subject and before the
body cease to function, they force some of their own blood into their future
childer. This makes the embrace of the Eerie the one most likely to fail as
subjects not all too seldom die before the embrace is completed. Often when
one of the Eerie embrace, he will call upon the assistance of a trusted
member of the Blessed caste, in case something goes wrong.
The turning usually takes half a day, senses are regained within seconds
after the first drop has entered the body and the screaming begin shortly
after, basic mobility regained after 4-5 hours and the pain start to fade away
after 10 hours.
The body of a killed Eerie has the look and appearance of a normal corpse.
Theories of it being possible to turn a corpse of the eerie to the Blessed, but
as of yet no successful turning is known.
When staked the Eerie take damage as per normal, as if there was no heart,
but having a bleed-out effect. They are strongly effected by sunlight, feeling
tired both mentally and physically, although when moving inside shadows
they feel no ill effect from the sun. They are able of creating ghouls, but
their blood are extremely addictive and have an rush effect, and if not
controlled correctly the ghouls can enter a murderous psychosis.
Creation Bonus:
Strength +2; Dexterity +2; Stamina +1Gangrel
Protectors of the wild, the voice of the forest, bringers of death.
Disciplines: Animalism or Obfuscate; Celerity or Vigor; Protean
Creation Bonus:
Strength +1; Resolve +2; Composure +2
Views:
• The Turned
Foul rituals and magic that should have been forgotten aeons ago,
they have forgotten their connection to the nature instead.
◦ Tremere
Of all foul ritualists, these are the most foul. But their magic will
not help their campaign.
◦ Ravnos
The right hand is not as foul as the left, but the connection has
been lost here to. Perhaps they could be made to rediscover that
connection.
• The Blessed
As the, as they claim, Blessed they see themselves as finer than us.
Which is fine, keep them and their ways away from me.
◦ Daeva
Arrogant bastards, you do know your not really human anymore,
right?
◦ Malkavian
If blessed mean mentally insane, then they are plenty blessed.
◦ Tzimisce
Thieves, we shall reclaim it eventually. You shall know the true
meaning of suffering before we are done.
◦ Ventrue
Always trying to use us as guard dogs, but we'll se who's the top
dog!
• The Eerie
We are true warriors, and we shall always be true.
◦ Nosferatu
Our comrade in arms. Though, they are scary as shit.
◦ Lasombra
I know the Lasombra, I would never go near them unless
properly invited. Never again.
• Humans
Most of them deserve no pity, if I see them around my woods they
better not do anything to disturb the peace.Nosferatu
The watchers, the spies, the assassins, the trader, the traitor.
Disciplines: Celerity or Obfuscate; Nightmare; Resilience or Vigor
Creation Bonus:
Dexterity +2; Wits +2; Intelligence +1
Views:
• The Turned
As different as night and day, one is human, the other is a monster.
And not as according to their tradition.
◦ Tremere
Ha ha ha ha, you think that..? Succeed, how? Hah! Seriously!
◦ Ravnos
I think their magic is actually more powerful then that of the
Tremere, I have a guy checking up on it.
• The Blessed
Not quite the blessing they make it out to be.
◦ Daeva
◦ Malkavian
◦ Tzimisce
◦ Ventrue
Always game for a good deal, good for me. And they believe I
do it for gold, they have yet to realise what they've lost.
• The Eerie
◦ Gangrel
They are nice, more brawn than brain, but still nice. Also, their
rituals are quite interesting.
◦ Lasombra
Sometimes, I believe they can see me creeping around in the
shadows. Still, they should know about shadows.
• Humans
Easily scared. They taste better then rats.. slightly better then rats,
though the quantities are in their favour.Lasombra
Officers and gentlemen, much less keen on killing you after inviting you in
to their home then before.
Disciplines: Auspex or Vigor; Dominate or Majesty; Obtenebration
Creation Bonus:
Manipulation +3; Strength +2
Views:
• The Turned
◦ Tremere
◦ Ravnos
• The Blessed
◦ Daeva
◦ Malkavian
◦ Tzimisce
◦ Ventrue
My thoughts of the Ventrue I dare not say, they seem to have
ears and eyes everywhere.
• The Eerie
◦ Gangrel
◦ Nosferatu
• Humans
Good sports; like cats, they jump at every shadow. Also, they make
for a good supper.Disciplines
Animalism
Although most look human, all the Kindred conceal within them a feral
predator, a Beast that divides all others into only two categories: threat or
prey. Some Kindred feel their affinity with the animals of the world, and
their connection with their own animalistic nature to a greater degree than
others. These Kindred often develop the Discipline of Animalism, which
allows them to bond with the beasts — and the Beasts — around them.
They can not only commune with lower creatures, but project their will
upon them, forcing them to obey. As the Kindred gain power, some develop
the ability to join with animals, or to influence the Beast lurking with their
own souls or the souls of other vampires.
Most Kindred are repellent to animals. Lesser creatures grow agitated in the
presence of the undead and normally flee the scene (or, in some cases,
attack the vampire in question). Kindred who possess Animalism are a very
different story. Animals are often attracted to such Kindred, and their
presence is soothing even to restless beasts.
Auspex
This potent Discipline grants a character superlative sensory capabilities. At
the lowest levels, it sharpens a Kindred’s mundane senses. As one
progresses in mastery, entirely new avenues of insight open up before the
user. Ultimately, this is the Discipline of gleaning information, whether that
data comes from sights and smells, from auras and patterns of energy or
directly from the mind of another creature. In addition, Auspex can be used
to pierce the veil of powers that cloud, dissemble and deceive (see the
“Clash of Wills” sidebar). Indeed, precious little can be kept secret from a
true master of Auspex. Once in a while, this uncanny Discipline provides
extrasensory and even precognitive visitations. Such premonitions might
come as quick flashes of imagery, overwhelming feelings of empathy or
even as an ominous sense of foreboding. The Kindred has absolutely no
control over these insights, but he can learn to interpret their significance
given time and experience. Such potent sensitivity can have its drawbacks,
however.
Celerity
Tales and legends of vampires ascribe to them inhuman speed, the ability to
move faster than the eye can see, and even to appear in two places at once.
While some of those accounts are exaggerated, Kindred with the Discipline
of Celerity can indeed move far faster than any mortal. They appear to blur
into nothingness, all others moving as if in slow motion in comparison.
Note that Celerity is obviously superhuman in use. Few Princes smile upon
uses of Celerity that leave too many curious mortal witnesses unaccounted
for.
Celerity is unlike many other Disciplines in that it is not actively rolled.
Rather, it provides a group of benefits, many of which affect other rolls.Chimerstry
The Ravnos are heirs to a legacy of illusion, and none can say exactly why.
The elders of their clan, when properly approached, speak cryptically of
ghuls and rakshasas, and the shapeshifting antics of their Antediluvian
founder are the subject of many a dark campfire tale among the clan. But
whatever the source, the nomadic Ravnos have a potent weapon in the form
of their Discipline of Chimerstry.
Chimerstry is an art of conjuration; the vampire may draw upon her inner
reserves to bring phantoms to life. These false images can confound mortal
senses and sensory equipment alike. If the Cainite's power is strong enough,
illusions created by Chimerstry may even baffle the heightened senses of
the vampire. The Ravnos are fond of using this power to seduce, swindle or
enslave mortals, effectively purchasing their victims' souls in exchange for a
sack of bouillon that isn't there.
Illusions created by Chimerstry may be detected by Auspex. They may also
be seen for what they are by a victim who ”proves” the illusion's falsehood.
Dementation
The special legacy of the Malkavian clan, Dementation allows the vampire
to channel madness, focus it, and pour it into the minds of those around
him.
The practitioner of Dementation need not actually be mad himself — at
least initially — although madness seems to grant a certain insight into the
key tenets of the Discipline. Few vampires ask the Malkavians to teach
them this Discipline, although the lunatics are almost always eager to
”enlighten” others. In fact, some say that one cannot learn the secrets of
Dementation without eventually going mad.
Eerily enough, Dementation doesn't seem to inflict insanity on its victims
per se. Rather, it seems to catalyse madness, breaking down doors into the
hidden reaches of the mind and releasing whatever it finds there. The
Malkavians claim that this is because insanity is the next step in the
evolution of the mind — a necessary progression if one is to behold the
truths of the universe. As such, they say, it is inherent to all minds, and
evident only in the more highly evolved specimens of human or vampiric
thought. Other Kindred pray the Malkavians are wrong, but find it difficult
to dismiss such thoughts out of hand, particularly because Dementation
works as well on vampires as it does on mortals...
Dominate
Some Kindred are capable of overwhelming the minds of others with their
own force of will, influencing actions and even thoughts. Use of Dominate
requires a character to capture a victim’s gaze. The Discipline can therefore
be used on only one subject at a time, and is useless if eye contact is not
possible.
Dominate does not grant the ability to make oneself understood or to
communicate mentally. Commands must be issued verbally, though certain
simple commands (such as “Go over there!” indicated with a pointed finger
and a forceful expression) may be conveyed by signs at the Storyteller’sdiscretion. No matter how powerful a vampire is, she cannot force her
victim to obey if she cannot make herself understood — if, for example, the
victim doesn’t speak the same language, she cannot hear or the orders
simply make no sense.
Note that victims of Dominate might realize what’s been done to them. That
is, they do not automatically sense that they are being controlled, but they
might subsequently wonder why they suddenly acted as they did. Wise
Kindred, especially those familiar with Dominate, are likely to figure it out
in the moment, and few vampires take kindly to being manipulated in such a
fashion. Most Kindred who develop Dominate are forceful, controlling
personalities, and they can make a reputation for themselves if they use this
Discipline wantonly.
Dominate is far more effective against mortals than it is against other
Kindred. Most Dominate abilities allow the victim to struggle against the
effects; that is, a contested roll is made against the Dominator. As no mortal
has Blood Potency, the vast majority of humans are at a disadvantage when
dealing with the Discipline.
Dominate is also more effective against those whom the user has subjected
to a Vinculum. A regnant may use some Dominate powers on a thrall
without the need for eye contact; the thrall merely has to hear the regnant’s
voice.
Majesty
One of the most legendary powers of the undead is the ability to attract,
sway and control the emotions of others, especially those of mortals.
Majesty is perhaps the most versatile of Disciplines, for its potential uses
and applications are both varied and multitudinous. The more savvy the
practitioner, the more use he can get out of the Discipline. Unlike some
other Disciplines, Majesty can be used on entire crowds of targets
simultaneously, making it even more potent — in the right hands. The only
requirement for use of most Majesty powers is that any potential targets see
the character. Eye contact is not required, nor is the ability to hear the
character (though it certainly doesn’t hurt).
The downside to Majesty, such as is it is, is that its subjects retain their free
will. Unlike victims of Dominate, who follow the commands of the Kindred
nearly mindlessly, those acting under Majesty are simply emotionally
predisposed to do whatever the power (or its user) suggests. While retention
of personality makes victims more useful in the long run, it also means they
require more care in handling than targets of Dominate. An abused victim of
Majesty certainly subverts or represses what his emotions suggest in order
to behave in the most appropriate manner. Meanwhile, subjects treated well
might be persuaded to act against even their own interests.
By and large, the Kindred who choose to develop their Majesty abilities are
those who recognize that one achieves more with honey than with vinegar.
Those who swear by Majesty often find Dominate, seen as “the flip side of
Majesty,” to be both boorish and crass, and they would swear to calling
upon it only in times of dire need.Nightmare
There’s no question that one of the foremost powers of legendary vampires
is the ability to strike fear in the hearts of mortal men. Also born of mortal
existence, other now-supernatural beings are susceptible. Fear is a fact of
existence that transcends any origins.
Vampires who delve into the dark side of their being — often exploring the
Beast or what it means to be monstrous — invest in the Discipline of
Nightmare. They learn to bear that which is terrifying or unholy about their
spirits, manifesting their inhumanity in their appearance or letting
unfortunate onlookers peer deep into the creatures’ depraved souls. The
results can take a jaded individual aback or subject an unsuspecting victim
to a fatal physiological reaction (to literally be frightened to death).
Practitioners of Nightmare explore this route to power for different reasons.
One vampire might exult in his inhuman nature and enjoy lording over
lessers. The Discipline offers immediate gratification, and these Kindred
display what is hideous about themselves to everyone, hiding it only insofar
as they must in order to observe the secrecy of the Traditions. Other undead
recognize the wisdom or even benevolence that fear affords. What better
way to deal with a problem or avoid a confrontation than by frightening
away an opponent? How better to protect someone from harm than by
scaring her off? And if one seeks solitude, striking fear is certainly more
effective than issuing threats, trying to reason with would-be intruders or
orchestrating ever more elaborate means by which to hide.
Obfuscate
Night-dwellers, predators by nature and keepers of the Masquerade,
vampires are inherently (and necessarily) creatures of secrecy and stealth.
From hiding minute objects to the ability to appear as someone else to the
power to fade from sight entirely, the Discipline of Obfuscate grants the
Kindred uncanny powers of concealment, stealth and deception.
Obfuscate clouds the mind in practice. For example, a character hiding an
object by using this Discipline doesn’t actually make the object disappear,
nor does someone using the Discipline to hide himself truly vanish. Rather,
the mind sees “around” the Obfuscated object, refusing to acknowledge it,
even if that requires a bit of filling in mental blanks. To continue the
example, if a character Obfuscated a large sheet of plywood and tried to
hide behind it herself, those looking at the plywood would, indeed, see the
character lurking behind it but not see the plywood itself.
The shroud of Obfuscate is very difficult to penetrate. Few Kindred or other
supernatural creatures can see through it, and only under the rarest of
circumstances do mortals have any hope. Because they operate on a less
conscious and mostly instinctual level, however, animals often perceive a
vampire’s presence — and react with appropriate fear or hostility — even if
they cannot detect him with their normal senses. Similarly, children, the
mentally ill and others who see the world in ways not quite normal might
pierce the deception.
Some Kindred with Auspex are able to see through Obfuscate, or at least
sense the presence of a supernatural deception.It’s important to note that Obfuscate affects the viewer’s mind, rather than
making any true physical change to the vampire. Thus, the Discipline is not
effective at cloaking a character from mechanical devices. Photographs,
video cameras and the like record the normal blurred image that all
vampires leave in such media, not the assumed appearance. Obfuscate does
affect any individual currently using the recording device, however, so
someone videotaping an Obfuscated vampire sees the illusion when looking
through the lens, discovering the truth only later when he reviews the tape
itself.
Obtenebration
The bailiwick of the Lasombra, the Obtenebration Discipline grants its user
power over darkness. The precise nature of the ”darkness” invoked is a
matter of debate among the Keepers. Some believe the power grants a
Kindred control over the stuff of her soul, allowing her to coax it tangibly
forth.
In any event, the effects of Obtenebration are terrifying, as waves of
enveloping blackness roil out from the vampire, washing over their targets
like an infernal tide. Blatant uses of this power are obvious breaches of the
Masquerade.
Vampires can see through the darkness they control, though other vampires
cannot. Dreadful tales of rival Keepers struggling to blind and smother each
other with the same wisps of darkness circulate among the young, though
no elders have come forth to substantiate these claims.
Protean
Easily one of the most overtly spectacular of the gifts of the Damned, the
Discipline of Protean is the study of physical metamorphosis and
transformation. The nature of this power is hotly debated among the
Kindred, for its abilities are so varied while simultaneously stemming from
no obvious aspect of the Curse. Whatever its cause or origin, Protean allows
its masters to assume virtually any form or shape.
Since the core of a vampire’s self doesn’t alter with his shape, a transformed
Kindred can generally take any action or use any Discipline that his new
form can reasonably allow. Gangrel in the form of a cloud of mist, for
example, could read auras (as the sense of sight doesn’t vanish), but
couldn’t Dominate someone effectively (as the prerequisite eye contact can
no longer be established). A vampire’s clothes and personal effects change
shape with him, but he cannot normally transmute especially large objects
or other creatures.
Protean powers — being permanent physical changes — last as long as the
vampire wishes them to, or until he is forced into torpor. Any state that
prevents the character from taking action (such as being staked) likewise
prevents transformation; the vampire needs the freedom to invoke his will.
Resilience
Legends abound of vampires who are able to withstand even the most brutal
punishment to their unliving forms. While all Kindred possess a certaindegree of the toughness of which these tales speak, those with the Discipline
of Resilience are commensurately more stalwart. Vampires with several dots
of Resilience are capable of walking through a hail of bullets, shrugging off
even the most punishing blows, and even resisting the deadly claws and
fangs of supernatural foes.
Resilience is unlike many other Disciplines in that it is not actively rolled.
Rather, it provides an augmentation of physical potential, which affects
other rolls.
Thaumaturgy
The Discipline of Thaumaturgy encompasses blood magic and other
sorcerous arts. Thaumaturgy is the unique possession of the Tremere and
one of its most jealously guarded secrets. Certain Kindred rumours even
speak of mystic cabals of Tremere that hunt down those thaumaturges who
are not members of the clan.
Thaumaturgy is a versatile and powerful Discipline. Its practice is divided
into two parts: paths and rituals. Thaumaturgical paths are applications of
the vampire's knowledge of blood magic, allowing her to create effects at
her whim. Rituals are more formulaic in nature, most akin to the ancient
magical “spell” of bygone nights. Because so many different paths and
rituals are available to the arcane Tremere, one never knows what to expect
when confronted with a practitioner of this Discipline.
Many vampires (wisely) fear the Discipline of Thaumaturgy. It is a very
potent and mutable Discipline, and almost anything the Kindred wishes may
be accomplished through its magic.
Vigor
Nearly every vampire legend across the globe expresses the preternatural
strength possessed by the undead. In truth, not all Kindred possess such
inhuman might, but the Discipline of Vigor makes those who do far more
powerful than any mortal could ever hope to be. Vigor allows Kindred to
strike opponents with the force of a falling boulder or speeding car; to lift
enormous weights as though they were paper; to shatter concrete like glass;
to leap distances so great that those elders with obscenely high levels of
Vigor may, in fact, be responsible for legends of vampiric flight.
Vigor is unlike many other Disciplines in that it is not actively rolled.
Rather, it provides an increase of physical strength, affecting other rolls.
Vicissitude
Vicissitude if the signature power of the Tzimisce and is almost unknown
outside the clan. Similar in some respects to Protean, Vicissitude allows the
Fiends to shape and sculpt their own or others' flash and bone. When a
Kindred uses Vicissitude to alter mortals, ghouls and vampires of lower
potency, the effects of the power are permanent. Naturally, a wielder can
always reshape her own flesh.
Note that while this Discipline permits powerful and horrific effects, the
wielder must obtain skin-to-skin contact and must often physically sculptthe desired result. This even applies to the use of the power on oneself.
Vampires skilled in Vicissitude are often inhumanly beautiful: those less
skilled are simply inhuman.
	

	
\end{document}